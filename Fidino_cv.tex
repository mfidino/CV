\documentclass[11pt,a4paper,nolmodern]{moderncv}

\usepackage{masonfidino}
\usepackage{fontawesome}
\usepackage[english]{babel}

\addto\captionsenglish{\renewcommand\refname{Publications \break \scriptsize \textcolor{black}{* denotes shared first authorship}}}





\linespread{0.9}
% for some reason, lines take up a lot of space in itemize in English...
\newenvironment{tightitemize}
   {\begin{itemize}
   \setlength{\parskip}{0pt}}
   {\end{itemize}}


% personal data
\title{Quantitative Ecologist}
\extrainfo{%
\faLinkedin\ \httplink{www.linkedin.com/in/mfidino}\\%
\faGithub\ \httplink{www.github.com/mfidino}\\%
} % optional, remove the line if not wanted

\begin{document}

\hyphenpenalty=10000
\maketitle

\section{Education}
\cventry {2013 -- 2017}{Ph.D.}{Ecology and Evolution}
{University of Illinois at Chicago}{}{Advisors: Joel Brown, Seth Magle, and Chris Whelan}
\cventry{2005 -- 2009}{B.S.}{Environmental Science}
{Western Washington University}{}{Advisor: David Wallin}

\section{Professional Experience}
\cventry{2017 -- Present}{Quantitative Ecologist}{Urban Wildlife Institute}{Lincoln Park Zoo}{Chicago, IL}{}
\cventry{2014 -- Present}{Analytics Advisor}{Urban Wildlife Information Network}{}{}{}
\cventry {2015 -- 2017}{Ecological Analyst}{Urban Wildlife Institute}{Lincoln Park Zoo}{Chicago, IL}{}
\cventry {2012 -- 2015}{Coordinator of Wildlife Management}{Urban Wildlife Institute}{Lincoln Park Zoo}{Chicago, IL}{ }
\cventry{2011 -- 2012}{Research Intern}{Urban Wildlife Institute}{Lincoln Park Zoo}{Chicago, IL}{ }
\cventry{2009 -- 2010}{Environmental Technician}{Environmental Assessment Services}{}{Richland, WA}{}

\nocite{*}

\bibliography{Fidino_publications}

\bibliographystyle{unsrt_abbrv}


\section{Selected Scientific Presentations}
\cvitem{2019} {A city's size and proportion of green space affects mammalian relative occupancy and response to urbanization: an analysis of 10 cities across the United States. \break The Internation Urban Wildlife Conference. Portland, Oregon.}
\cvitem{2018}{Advancing urban wildlife knowledge through a multi-city collaboration. \break The Wildlife Society. Cincinnati, Ohio.}
\cvitem{---}{Long-term declines of a highly interactive species. Society for Conservation Biology. Toronto, Ontario. }
\cvitem{---}{Long-term declines of a highly interactive species. International Association for Landscape Ecology. Chicago, Illinois. }
\cvitem{2017}{Using Fourier series to predict periodic patterns in dynamic occupancy models. \break Ecological Society of America. Portland, Oregon.}
\cvitem{---}{Quantifying the structural and functional connectivity of habitat patches for Chicago area mesocarnivores. International Urban Wildlife Conference. San Diego, California.}
\cvitem{2016}{A Bayesian approach to incorporate patterns of co-occurrence into multi-species occupancy models. Society for Conservation Biology. Madison, Wisconsin.}
\cvitem{2015}{Mesocarnivore dynamics in a highly fragmented, yet highly permeable urban landscape. Ecological Society of America. Baltimore, Maryland.}
\cvitem{2014}{Habitat dynamics of the Virginia opossum (\textit{Didelphis virginiana}) in a highly urban landscape. The Wildlife Society. Pittsburg, Pennsylvania.}

\section{Selected Invited Presentations}
\cvitem{2020}{\textbf{Lecture:} Urban wildlife research in Chicago and beyond. Loyola University. Chicago, Illinois.}
\cvitem{2019}{\textbf{Plenary:} Harnessing UWIN data to reshape the future of cities. The Urban Wildlife Information Network summit. Chicago, Illinois.}
\cvitem{---}{\textbf{Lecture:} A city's size and proportion of green space affects mammalian relative occupancy and response to urbanization. Texas Tech University. Lubbock, Texas.}
\cvitem{---}{\textbf{Seminar series:}  Camera trapping across North America. The Fort Dearborn Audubon Society. Chicago, Illinois.}
\cvitem{2018}{\textbf{Lecture:} Advancing urban wildlife knowledge through a multi-city collaboration. \break International workshop on biodiversity and the urban-rural interface. Linde, Germany.}
\cvitem{---}{\textbf{Seminar series:} Urban wildlife through space and time. Seminar series at Butler University. Indianapolis, Indiana.}
\cvitem{2016}{\textbf{Seminar series:} A historical analysis of bird species diversity in Lincoln Park, Chicago during spring migration. The Fort Dearborn Audubon Society. Chicago, Illinois.}



\section{Grants and Awards}
\cvitem{2020 -- 2022}{\textbf{NSF} -- Impacts of Urban Rats and Rodent Control on Public Health and Urban Wildlife Conservation (Senior Personnel, Award \#1923882, \$680,466).} 
\cvitem{2018 -- 2020}{\textbf{Grainger Foundation} -- Urban Wildlife Information Network expansion (Co-PI, \$250,000).}
\cvitem{2014 -- 2020}{\textbf{Abra Prentice Wilkins Foundation} -- Urban Wildlife Institute Expansion (Co-PI, \$1,500,000).}
\cvitem{2014}{\textbf{The American Bluebird Society} -- Assessing the nest success of urban cavity nesting birds (Co-PI, \$600).}

\section{Teaching Experience}
\cvitem{2020}{\textbf{Mammal ID lab:} Presented on Illinois mammal identification tips for an undergraduate course at Loyola University.}
\cvitem{2019}{\textbf{Occupancy modeling and data management:} Gave a short course on the basic assumptions of occupancy modeling and camera trapping data management to 40 researchers at the Urban Wildlife Information Network summit held at the Lincoln Park Zoo.}
\cvitem{2018}{\textbf{Occupancy modeling in R:} Taught R programming and occupancy modeling to students and faculty at Butler University in Indianapolis, Indiana.}
\cvitem{2017}{\textbf{Software Carpentry course on R programming:} Assisted with course held at University of Illinois at Chicago.}
\cvitem{2016 -- Present}{\textbf{R programming and occupancy modeling:} Developed a two-day workshop to teach students, faculty and new partners to the Urban Wildlife Information Network the basics of R programming and how to model detection/non-detection data collected via camera trapping.}
\cvitem{2016}{\textbf{Workshop on generalized linear models, power analysis, and simulations in R:} Developed workshop to teach Lincoln Park Zoo staff on basics of generalized linear models and how to simulate data in R.}
\cvitem{2014}{\textbf{A review of bird count methods and analyses:} A workshop I developed and presented to the Chicago Audubon Society}.


\section{Reviewer}
\cvitem{}{\textit{Journal count}: 18}
\cvitem{}{Biological Conservation, Canadian Journal of Zoology, Ecography, Ecology and Evolution, Ecological Applications, Ecology Letters, Environmental Conservation, Environmental Management, Human Dimensions of Wildlife, Journal of Fish and Wildlife Management, Journal of Mammalogy, Journal of Ornithology, Landscape and Urban Planning, PLOS One, Proceedings of the Royal Society B, Urban Ecosystems, The Wildlife Society Bulletin, Zoo Biology}


\section{Committee member}
\cvitem{2020 -- Present}{\textbf{External Ph.D. committee member} for Katie Fowler at University of Illinois at Chicago.\break Thesis topic: Stress responses in captive and wild mammal populations.}
\cvitem{2020 -- Present}{\textbf{External Ph.D. committee member} for Rachel Larson at University of Iowa. \break Thesis topic: Trophic interactions in urban environments.}
\cvitem{2018 -- Present}{\textbf{External Ph.D. committee member} for Anna Kase at University of South Dakota. \break Thesis topic: False map turtle (\textit{Graptemys pseudogeographica}) abundance and habitat utilization in the Missouri River, South Dakota.  }

\section{Service and Outreach}
\cvitem{}{\textbf{Accessibility working group member} at the Lincoln Park Zoo, which is tasked to make programs at the zoo more accessible for people with disabilities.} 
\cvitem{}{\textbf{Environmental Justice working group member} for the Urban Wildlife Information Network.} 
\cvitem{}{\textbf{Moderator on Chicago Wildlife Watch}, which is a citizen science project for people to help classify the camera trap images we collect throughout Chicago.}
\cvitem{}{\textbf{Research committee member} at the Lincoln Park Zoo, which is  tasked to evaluate research proposals from external and internal researchers who wish to conduct research on zoo grounds, with zoo data, or with zoo resources.} 
\cvitem{}{\textbf{Technology working group member} for the Urban Wildlife Information Network.} 
\section{Academic Organizations}
\cvitem{2015 -- Present}{Ecological Society of America}
\cvitem{2017 -- Present}{Society for Conservation Biology}
\cvitem{2014 -- Present}{The Wildlife Society}









\end{document}