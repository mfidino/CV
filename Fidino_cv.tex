\documentclass[11pt,a4paper,nolmodern]{moderncv}

\usepackage{masonfidino}
\usepackage{fontawesome}
\usepackage[english]{babel}

\addto\captionsenglish{\renewcommand\refname{Publications \break \scriptsize \textcolor{black}{* denotes shared first authorship}}}




\linespread{0.9}
% for some reason, lines take up a lot of space in itemize in English...
\newenvironment{tightitemize}
   {\begin{itemize}
   \setlength{\parskip}{0pt}}
   {\end{itemize}}


% personal data
\title{Quantitative Ecologist}
\extrainfo{%
\faLinkedin\ \httplink{www.linkedin.com/in/mfidino}\\%
\faGithub\ \httplink{www.github.com/mfidino}\\%
} % optional, remove the line if not wanted

\begin{document}


\hyphenpenalty=10000
\maketitle

\section{Education}
\cventry {2013 -- 2017}{Ph.D.}{Ecology and Evolution}
{University of Illinois at Chicago}{}{Advisors: Joel Brown, Seth Magle, and Chris Whelan}
\cventry{2005 -- 2009}{B.S.}{Environmental Science}
{Western Washington University}{}{Advisor: David Wallin}

\section{Professional Experience}
\cventry{2017 -- Present}{Quantitative Ecologist}{Urban Wildlife Institute}{Lincoln Park Zoo}{Chicago, IL}{In this role I closely collaborate with conservation and science staff at the Lincoln Park Zoo to better manage and analyze data from a wide variety of sources, from touchscreen cognition tests for primates to nation-wide camera trapping projects. My own research in this position primarily focuses on developing new tools and techniques that better incorporate a species natural history into a statistical model and help address basic and applied questions related to how species respond to habitat fragmentation.}
\cventry {2015 -- 2017}{Ecological Analyst}{Urban Wildlife Institute}{Lincoln Park Zoo}{Chicago, IL}{In this role I worked to develop techniques to better manage and analyze ecological data and assist other science centers across the Lincoln Park Zoo with statistical analyses, data management, and computer programming. I also developed statistics and computer programming workshops for staff. Finally, I helped initiate a large-scale citizen science project, Chicago Wildlife Watch (\httplink{www.chicagowildlifewatch.org}) and have written extensive software to verify and upload data to the project.}
\cventry{2014 -- Present}{Analytics Advisor}{Urban Wildlife Information Network}{}{}{The Urban Wildlife Information Network (UWIN) is the world's first systematic multi-city urban wildlife monitoring network. To facilitate the collection and analysis of these data I worked closely with a tech company in Chicago to develop a website and SQL datatbase to store camera trapping images, identify species in images, and summarize the data for varying analyses. Furthermore, I work with partners in over 17 cities to answer any database or analysis question they may have.}
\cventry {2012 -- 2015}{Coordinator of Wildlife Management}{Urban Wildlife Institute}{Lincoln Park Zoo}{Chicago, IL}{Led research of wildlife on zoo grounds, which included daily avian point counts, tracking relocated turtles in a newly restored pond habitat, on grounds rabbit management, arthropod surveys, and monitoring a nesting colony of state-endangered Black-crowned Night Herons (\textit{Nycticorax nycticorax}). To do all of this work, I managed and trained teams of interns each year, mentored them on urban ecology, and helped them present their work to their peers at the Lincoln Park Zoo. Additionally, I collaborated extensively with the Lincoln Park Zoo's education department on numerous projects to educate zoo visitors on wildlife conservation efforts throughout Chicago. }
\cventry{2011 -- 2012}{Research Intern}{Urban Wildlife Institute}{Lincoln Park Zoo}{Chicago, IL}{Aided with field work and data entry for the Urban Wildlife Institute's biodiversity monitoring survey, helped with research on zoo grounds, and created year-end permit reports. }
\cventry{2009 -- 2010}{Environmental Technician}{Environmental Assessment Services}{}{Richland, WA}{Assisted with numerous biological research projects at the Hanford superfund site.}

\nocite{*}

\bibliography{Fidino_publications}

\bibliographystyle{unsrt_abbrv}


\section{Selected Scientific Presentations}
\cvitem{2019} {A city's size and proportion of green space affects mammalian relative occupancy and response to urbanization: an analysis of 10 cities across the United States. \break The Internation Urban Wildlife Conference. Portland, Oregon.}
\cvitem{2018}{Advancing urban wildlife knowledge through a multi-city collaboration. \break The Wildlife Society. Cincinnati, Ohio.}
\cvitem{---}{Long-term declines of a highly interactive species. Society for Conservation Biology. Toronto, Ontario. }
\cvitem{---}{Long-term declines of a highly interactive species. International Association for Landscape Ecology. Chicago, Illinois. }
\cvitem{2017}{Using Fourier series to predict periodic patterns in dynamic occupancy models. \break Ecological Society of America. Portland, Oregon.}
\cvitem{---}{Quantifying the structural and functional connectivity of habitat patches for Chicago area mesocarnivores. International Urban Wildlife Conference. San Diego, California.}
\cvitem{2016}{A Bayesian approach to incorporate patterns of co-occurrence into multi-species occupancy models. Society for Conservation Biology. Madison, Wisconsin.}
\cvitem{2015}{Mesocarnivore dynamics in a highly fragmented, yet highly permeable urban landscape. Ecological Society of America. Baltimore, Maryland.}
\cvitem{2014}{Habitat dynamics of the Virginia opossum (\textit{Didelphis virginiana}) in a highly urban landscape. The Wildlife Society. Pittsburg, Pennsylvania.}

\section{Selected Invited Presentations}
\cvitem{2019}{Harnessing UWIN data to reshape the future of cities. Plenary talk at the Urban Wildlife Information Network summit. Chicago, Illinois.}
\cvitem{2018}{Advancing urban wildlife knowledge through a multi-city collaboration. \break International workshop on biodiveristy and the urban-rural interface. Linde, Germany.}
\cvitem{---}{Urban wildlife through space and time. Seminar series at Butler University. Indianapolis, Indiana.}
\cvitem{2016}{A historical analysis of bird species diversity in Lincoln Park, Chicago during spring migration. Seminar series for the Fort Dearborn Audubon Society. Chicago, Illinois.}
\cvitem{2014}{A review of bird count methods. Chicago Audubon Society. Chicago, Illinois.}



\section{Grants and Awards}
\cvitem{2020 -- 2022}{\textbf{NSF} -- Impacts of Urban Rats and Rodent Control on Public Health and Urban Wildlife Conservation (Senior Personnel, \$680,466).} 
\cvitem{2018 -- Present}{\textbf{Grainger} -- Urban Wildlife Information Network expansion (Co-PI, \$250,000).}
\cvitem{2014 -- Present}{\textbf{Abra Prentice Wilkins} -- Urban Wildlife Institute Expansion (Co-PI, \$1,500,000).}
\cvitem{2014}{\textbf{The American Bluebird Society} -- Assessing the nest success of urban cavity nesting birds (Co-PI, \$600).}

\section{Teaching Experience}
\cvitem{2019}{\textbf{Occupancy modeling and data management:} Gave a short course on the basic assumptions of occupancy modeling and camera trapping data management to 40 researchers at the Urban Wildlife Information Network summit held at the Lincoln Park Zoo.}
\cvitem{2018}{\textbf{Occupancy modeling in R:} Invited to teach R and occupancy modeling to students and faculty at Butler University in Indianapolis, Indiana.}
\cvitem{2017}{\textbf{Software Carpentry course on R programming:} Assisted with course held at University of Illinois at Chicago.}
\cvitem{2016 -- Present}{\textbf{R programming and occupancy modeling:} Developed a two-day workshop to teach students, faculty and new partners to the Urban Wildlife Information Network the basics of R programming and how to model detection/non-detection data collected via camera trapping.}
\cvitem{2016}{\textbf{Workshop on generalized linear models, power analysis, and simulations in R:} Developed workshop to teach Lincoln Park Zoo staff on basics of generalized linear models and how to simulate data in R.}


\section{Reviewer}
\cvitem{}{Biological Conservation, Canadian Journal of Zoology, Ecography, Ecology and Evolution, Ecological Applications, Ecology Letters, Human Dimensions of Wildlife, Journal of Fish and Wildlife Management, Journal of Mammalogy, Urban Ecosystems, The Wildlife Society Bulletin}


\section{Service and Outreach}
\cvitem{}{\textbf{Ph.D. Committee member} for Anna Kase at University of South Dakota. \break Thesis topic: False map turtle (\textit{Graptemys pseudogeographica}) abundance and habitat utilization in the Missouri River, South Dakota.  }
\cvitem{}{\textbf{Research committee member} at the Lincoln Park Zoo, which is  tasked to evaluate research proposals from external and internal researchers who wish to conduct research on zoo grounds, with zoo data, or with zoo resources.} 
\cvitem{}{\textbf{Accessibility working group member} at the Lincoln Park Zoo, which is tasked to make programs at the zoo more accessible for people with disabilities.} 
\cvitem{}{\textbf{Moderator on Chicago Wildlife Watch}, which is a citizen science project for people to help classify the camera trap images we collect throughout Chicago.}
\section{Academic Organizations}
\cvitem{2015 -- Present}{Ecological Society of America}
\cvitem{2017 -- Present}{Society for Conservation Biology}
\cvitem{2014 -- Present}{The Wildlife Society}









\end{document}