\documentclass[11pt,a4paper,nolmodern]{moderncv}

\usepackage{masonfidino}
\usepackage{fontawesome}
\usepackage[english]{babel}
\usepackage{multicol}
\usepackage{enumitem}

\addto\captionsenglish{\renewcommand\refname{Publications \break \scriptsize \textcolor{black}{* denotes shared first authorship}}}


\linespread{0.9}
% for some reason, lines take up a lot of space in itemize in English...
\newenvironment{tightitemize}
   {\begin{itemize}
   \setlength{\parskip}{0pt}}
   {\end{itemize}}


% personal data
\title{Quantitative Ecologist}
\extrainfo{%
\faLinkedin\ \httplink{www.linkedin.com/in/mfidino}\\%
\faGithub\ \httplink{www.github.com/mfidino}\\%
} % optional, remove the line if not wanted

\begin{document}

\hyphenpenalty=10000
\maketitle

\section{Education}
\cventry {2013 -- 2017}{Ph.D.}{Ecology and Evolution}
{University of Illinois at Chicago}{}{Advisors: Joel Brown, Seth Magle, and Chris Whelan}
\cventry{2005 -- 2009}{B.S.}{Environmental Science}
{Western Washington University}{}{Advisor: David Wallin}

\section{Professional Experience}
\cventry{2024 -- Present}{Senior Quantitative Ecologist}{Urban Wildlife Institute}{Lincoln Park Zoo}{Chicago, IL}{}
\cventry{2023 -- Present}{Faculty}{Committee on Evolutionary Biology}{University of Chicago}{Chicago, IL}{}
\cventry{2017 -- 2024}{Quantitative Ecologist}{Urban Wildlife Institute}{Lincoln Park Zoo}{Chicago, IL}{}
\cventry {2015 -- 2017}{Ecological Analyst}{Urban Wildlife Institute}{Lincoln Park Zoo}{Chicago, IL}{}
\cventry {2012 -- 2015}{Coordinator of Wildlife Management}{Urban Wildlife Institute}{Lincoln Park Zoo}{Chicago, IL}{ }
\cventry{2011 -- 2012}{Research Intern}{Urban Wildlife Institute}{Lincoln Park Zoo}{Chicago, IL}{ }
\cventry{2009 -- 2010}{Environmental Technician}{Environmental Assessment Services}{}{Richland, WA}{}

\section{Honorary Appointments}
\cventry{2023 -- Present}{Research Associate}{Integrative Research Center}{Field Museum}{Chicago, IL}{}

\nocite{*}

\bibliography{Fidino_pubs2}


\bibliographystyle{unsrt_abbrv}


\section{Selected Scientific Presentations}
\cvitem{2025}{Leveraging large-scale research networks to generate near-term spatial forecasts of wildlife distributions in unsampled regions. The International Urban Wildlife Conference. Atlanta, Georgia.}
\cvitem{2024}{Gentrification drives patterns of alpha and beta diversity in cities. The Wildlife Society Conference. Baltimore, MD.}
\cvitem{---}{Leveraging large-scale research networks to generate near-term spatial forecasts of wildlife distributions in unsampled regions. The Wildlife Society Conference. Baltimore, MD.}
\cvitem{2023}{Gentrification drives patterns of alpha and beta diversity across American cities. The International Urban Wildlife Conference. Washington, D.C.}
\cvitem{2022}{Integrated species distribution models reveal spatiotemporal patterns of human-wildlife conflict. The Wildlife Society. Spokane, Washington.}
\cvitem{2021}{Rethinking mammal habitat occupancy modeling and the role of diel activity in an anthropogenic world.  The Wildlife Society. Virtual.}
\cvitem{---}{Rethinking mammal habitat occupancy modeling and the role of diel activity in an anthropogenic world.  The International Urban Wildlife conference. Virtual.}
\cvitem{2019} {A city's size and proportion of green space affects mammalian relative occupancy and response to urbanization: an analysis of 10 cities across the United States. \break The International Urban Wildlife Conference. Portland, Oregon.}
\cvitem{2018}{Advancing urban wildlife knowledge through a multi-city collaboration. \break The Wildlife Society. Cincinnati, Ohio.}
\cvitem{---}{Long-term declines of a highly interactive species. Society for Conservation Biology. Toronto, Ontario. }
\cvitem{---}{Long-term declines of a highly interactive species. International Association for Landscape Ecology. Chicago, Illinois. }
\cvitem{2017}{Using Fourier series to predict periodic patterns in dynamic occupancy models. \break Ecological Society of America. Portland, Oregon.}
\cvitem{---}{Quantifying the structural and functional connectivity of habitat patches for Chicago area mesocarnivores. International Urban Wildlife Conference. San Diego, California.}
\cvitem{2016}{A Bayesian approach to incorporate patterns of co-occurrence into multi-species occupancy models. Society for Conservation Biology. Madison, Wisconsin.}
\cvitem{2015}{Mesocarnivore dynamics in a highly fragmented, yet highly permeable urban landscape. Ecological Society of America. Baltimore, Maryland.}
\cvitem{2014}{Habitat dynamics of the Virginia opossum (\textit{Didelphis virginiana}) in a highly urban landscape. The Wildlife Society. Pittsburg, Pennsylvania.}

\section{Conference Symposia Organized}
\cvitem{2025}{The Urban Wildlife Information Network. The International Urban Wildlife Conference. Atlanta, Georgia}
\cvitem{2023}{The Urban Wildlife Information Network: a research alliance to increase our understanding of urban environments from local to global scales. The International Urban Wildlife Conference. Washington, D.C.}

\section{Selected Invited Presentations}
\cvitem{2025}{\textbf{Sherwood Ebey Lecture Series:} Scaling up urban biodiversity monitoring with coordinated research networks. Sewanee. Sewanee, Tennessee}
\cvitem{---}{\textbf{Sherwood Ebey Lecture Series:} Recent advancements in quantifying and categorizing how animals use diel time. Sewanee. Sewanee, Tennessee}
\cvitem{2024}{\textbf{Seminar series:} Scaling up urban biodiversity monitoring with coordinated research networks. University of Washington. Seattle, Washington.}
\cvitem{---}{\textbf{Lecture:} The Urban Wildlife Information Network: a research alliance to increase our understanding of urban environments from local to global scales. Openlands. Chicago, Illinois.}
\cvitem{---}{\textbf{Lecture:} Gentrification drives patterns of alpha and beta diversity across US Cities. Mayors Committee for Wildlife. Chicago, Illinois.}
\cvitem{---}{\textbf{Lecture:} The Urban Wildlife Information Network. The Global Wildlife Data Sharing Conference. Richland, Washington.}
\cvitem{---}{\textbf{Seminar series:} The Urban Wildlife Information Network: a research alliance to increase our understanding of urban environments from local to global scales. University of Minnesota. St. Paul, Minnesota.}
\cvitem{---}{\textbf{Keynote speaker:} The Urban Wildlife Information Network: a research alliance to increase our understanding of urban environments from local to global scales. The Urban Ecosystem Research Consortium. Portland, Oregon.}
\cvitem{---}{\textbf{Lecture:} Defining mammalian diel activity and plasticity.  Universidad Nacional del Comahue. Bariloche, Argentina. }
\cvitem{2023}{\textbf{Seminar series:} Scaling up urban ecology through a global camera trap study.  University of Wisconsin-Madison. Madison, Wisconsin. }
\cvitem{---}{\textbf{Seminar series:} Scaling up urban ecology through a global camera trap study. Watson Armour seminar series. Field Museum. Chicago, Illinois. }
\cvitem{---}{\textbf{Keynote speaker:} Disentangling spatiotemporal variation in mammaliam responses to urbanization and diel activity patterns through a global camera trap study. Michigan State University Ecology \& Evolutionary Biology Research Symposium. East Lansing, Michigan.}
\cvitem{---}{\textbf{Seminar series:} Disentangling spatiotemporal variation in mammaliam responses to urbanization and diel activity patterns through a global camera trap study. University of Chicago. Chicago, Illinois.}
\cvitem{2022}{\textbf{Seminar series:} Teasing apart among and within city variation in urban biodiversity through a large-scale, multi-city collaboration. University of Chicago. Chicago, Illinois.}
\cvitem{2021}{\textbf{Seminar series:} Teasing apart among and within city variation in urban biodiversity through a large-scale, multi-city collaboration. University of Nebraska Lincoln. Lincoln, Nebraska.}
\cvitem{---}{\textbf{Seminar series:} Teasing apart among and within city variation in urban biodiversity through a large-scale, multi-city collaboration. University of Rhode Island. South Kingstown, Rhode Island.}
\cvitem{---}{\textbf{Seminar series:} Teasing apart among and within city variation in urban biodiversity through a large-scale, multi-city collaboration. University of Zurich. Zurich, Switzerland.}
\cvitem{2020}{\textbf{Lecture:} Urban wildlife research in Chicago and beyond. Loyola University. Chicago, Illinois.}
\cvitem{2019}{\textbf{Plenary:} Harnessing UWIN data to reshape the future of cities. The Urban Wildlife Information Network summit. Chicago, Illinois.}
\cvitem{---}{\textbf{Lecture:} A city's size and proportion of green space affects mammalian relative occupancy and response to urbanization. Texas Tech University. Lubbock, Texas.}
\cvitem{---}{\textbf{Seminar series:}  Camera trapping across North America. The Fort Dearborn Audubon Society. Chicago, Illinois.}
\cvitem{2018}{\textbf{Lecture:} Advancing urban wildlife knowledge through a multi-city collaboration. \break International workshop on biodiversity and the urban-rural interface. Linde, Germany.}
\cvitem{---}{\textbf{Seminar series:} Urban wildlife through space and time. Seminar series at Butler University. Indianapolis, Indiana.}
\cvitem{2016}{\textbf{Seminar series:} A historical analysis of bird species diversity in Lincoln Park, Chicago during spring migration. The Fort Dearborn Audubon Society. Chicago, Illinois.}



\section{Grants and Awards}
\cvitem{}{\textit{Current total}: 6.12 million dollars, US.}
\cvitem{2025 -- 2027}{\textbf{Walder Foundation} -- Linking Avian Migratory Stopover to Legacies of Urban Inequality (PI, \$228,488).}
\cvitem{2023 -- 2028}{\textbf{NSF} -- DISES: Social-ecological drivers and consequences of human-carnivore interactions within and among American cities (Co-PI, Award \#2307324, \$1,595,162).}
\cvitem{2023}{\textbf{Pariveda Solutions} -- Incorporating Acoustic Recording Units and a Machine Learning Pipeline to Identify Birds by Song to the Urban Wildlife Information Network Database (PI, \$400,000).}
\cvitem{2023}{\textbf{Walder Foundation} -- A Global Expansion of the Urban Wildlife Information Network (Co-PI, \$1,000,000).}
\cvitem{2022 -- 2023}{\textbf{Theodore Roosevelt Genius Prize Competition Promotion of Wildlife Conservation} --  Urban Wildlife and Me: Harnessing machine learning to connect urban residents to wildlife conservation through social media (Senior Personnel, \$100,000).}
\cvitem{2020 -- 2022}{\textbf{NSF} -- Impacts of Urban Rats and Rodent Control on Public Health and Urban Wildlife Conservation (Senior Personnel, Award \#1923882, \$680,466).} 
\cvitem{2018}{\textbf{Pariveda Solutions} -- Creating a Web Application to Store, Annotate, and Report Camera Trap Data for the Urban Wildlife Information Network (Co-PI, \$360,000).}
\cvitem{2018 -- 2020}{\textbf{Grainger Foundation} -- Urban Wildlife Information Network expansion (Co-PI, \$250,000).}
\cvitem{2014 -- 2020}{\textbf{Abra Prentice Wilkins Foundation} -- Urban Wildlife Institute Expansion (Co-PI, \$1,500,000).}
\cvitem{2014}{\textbf{The American Bluebird Society} -- Assessing the nest success of urban cavity nesting birds (Co-PI, \$600).}

\section{Teaching Experience}
\cvitem{2025}{\textbf{Workshop on camera trapping, reproducible science, and statistical modeling:} Developed and ran two day workshop on camera trap study design, reproducible science, and occupancy modeling to Urban Wildlife Information Network partners in San Carlos de Bariloche, Argentina.}
\cvitem{2024}{\textbf{Workshop on advanced topics in occupancy modeling:} Developed and ran workshop for the Urban Wildlife Information Network virtual workshop series.}
\cvitem{---}{\textbf{Workshop on GitHub and data reproducibility:} Developed and ran workshop for the Urban Wildlife Information Network virtual workshop series.}
\cvitem{2022}{\textbf{Lecture:} Presented on the luxury effect and how systemic racism can structure biodiversity in cities at Northwestern University.}
\cvitem{2021}{\textbf{Lecture:} Presented on the luxury effect and how systemic racism can structure biodiversity in cities at University of Rhode Island.}
\cvitem{2020}{\textbf{Mammal ID lab:} Presented on Illinois mammal identification tips for an undergraduate course at Loyola University.}
\cvitem{2019}{\textbf{Occupancy modeling and data management:} Gave a short course on the basic assumptions of occupancy modeling and camera trapping data management to 40 researchers at the Urban Wildlife Information Network summit held at the Lincoln Park Zoo.}
\cvitem{2018}{\textbf{Occupancy modeling in R:} Taught R programming and occupancy modeling to students and faculty at Butler University in Indianapolis, Indiana.}
\cvitem{2017}{\textbf{Software Carpentry course on R programming:} Assisted with course held at University of Illinois at Chicago.}
\cvitem{2016}{\textbf{R programming and occupancy modeling:} Developed a two-day workshop to teach students, faculty and new partners to the Urban Wildlife Information Network the basics of R programming and how to model detection/non-detection data collected via camera trapping.}
\cvitem{---}{\textbf{Workshop on generalized linear models, power analysis, and simulations in R:} Developed workshop to teach Lincoln Park Zoo staff on basics of generalized linear models and how to simulate data in R.}
\cvitem{2014}{\textbf{A review of bird count methods and analyses:} A workshop I developed and presented to the Chicago Audubon Society}.

\section{Adjunct faculty}
\cvitem{2022 -- 2023}{University of South Dakota.}

\section{Reviewer}
\cvitem{}{\textit{Journal count}: 42}
\cvitem{}{\textit{Manuscripts reviewed per year since 2020}: 2020 (12), 2021 (17), 2022 (28), 2023 (19), 2024 (19), 2025 (8)}


\cvitem{}{
  \textit{Journals reviewed for:}
  \begin{multicols}{2}
  \begin{itemize}[leftmargin=*, noitemsep]
    \item Biological Conservation
    \item Biotropica
    \item Canadian Journal of Zoology
    \item Conservation Science and Practice
    \item Ecography
    \item Ecology
    \item Ecology and Evolution
    \item Ecological Applications
    \item Ecological Indicators
    \item Ecology Letters
    \item Ecosphere
    \item Environmental Conservation
    \item Environmental Management
    \item Frontiers in Ecology and the Environment
    \item Global Change Biology
    \item Human Dimensions of Wildlife
    \item Human–Wildlife Interactions
    \item Journal of Agricultural Biological and Environmental Statistics
    \item Journal of Animal Ecology
    \item Journal of Applied Ecology
    \item Journal of Applied Statistics
    \item Journal of Biogeography
    \item Journal of Fish and Wildlife Management
    \item Journal of Wildlife Management
    \item Journal of Mammalogy
    \item Journal of Ornithology
    \item Journal of Urban Ecology
    \item Landscape and Urban Planning
    \item Methods in Ecology and Evolution
    \item Nature Communications
    \item Nature Ecology \& Evolution
    \item Oikos
    \item PCI Ecology
    \item PeerJ
    \item PLOS ONE
    \item Proceedings of the Royal Society B
    \item Royal Society Open Science
    \item Trees, Forests \& People
    \item Urban Ecosystems
    \item Wildlife Research
    \item Wildlife Society Bulletin
    \item Zoo Biology
  \end{itemize}
  \end{multicols}
}



\section{Committee member}
\cvitem{2024 -- Present}{\textbf{External masters committee member} for Alexandria Hiott at Texas A\&M University.\break Thesis topic: Understanding disease transmission dynamics among domestic and wild felid species.}
\cvitem{2020 -- 2024}{\textbf{External Ph.D. committee member} for Katie Fowler at University of Illinois at Chicago.\break Thesis topic: Stress responses in captive and wild mammal populations.}
\cvitem{2020 -- 2024}{\textbf{External Ph.D. committee member} for Rachel Larson at University of Iowa. \break Thesis topic: Trophic interactions in urban environments.}
\cvitem{2021 -- 2023}{\textbf{External Ph.D. committee member} for Krista Shires at George Mason University.\break Thesis topic: The effect of urbanization on urban adapter species.}
\cvitem{2018 -- 2022}{\textbf{External Ph.D. committee member} for Anna Kase at University of South Dakota. \break Thesis topic: False map turtle (\textit{Graptemys pseudogeographica}) abundance and habitat utilization in the Missouri River, South Dakota.  }

\section{Service and Outreach}
\cvitem{}{\textbf{Analytics Advisor} for the Urban Wildlife Information Network.}
\cvitem{}{\textbf{Accessibility working group member} at the Lincoln Park Zoo, which is tasked to make programs at the zoo more accessible for people with disabilities.} 
\cvitem{}{\textbf{Environmental Justice working group member} for the Urban Wildlife Information Network.} 
\cvitem{}{\textbf{Moderator on Chicago Wildlife Watch}, which is a citizen science project for people to help classify the camera trap images we collect throughout Chicago.}
\cvitem{}{\textbf{Research committee member} at the Lincoln Park Zoo, which is  tasked to evaluate research proposals from external and internal researchers who wish to conduct research on zoo grounds, with zoo data, or with zoo resources.} 
\cvitem{}{\textbf{Technology working group member} for the Urban Wildlife Information Network.} 
\section{Academic Organizations}
\cvitem{2015 -- Present}{Ecological Society of America}
\cvitem{2017 -- Present}{Society for Conservation Biology}
\cvitem{2014 -- Present}{The Wildlife Society}









\end{document}